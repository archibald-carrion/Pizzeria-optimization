% Options for packages loaded elsewhere
\PassOptionsToPackage{unicode}{hyperref}
\PassOptionsToPackage{hyphens}{url}
\documentclass[
]{article}
\usepackage{xcolor}
\usepackage[margin=1in]{geometry}
\usepackage{amsmath,amssymb}
\setcounter{secnumdepth}{-\maxdimen} % remove section numbering
\usepackage{iftex}
\ifPDFTeX
  \usepackage[T1]{fontenc}
  \usepackage[utf8]{inputenc}
  \usepackage{textcomp} % provide euro and other symbols
\else % if luatex or xetex
  \usepackage{unicode-math} % this also loads fontspec
  \defaultfontfeatures{Scale=MatchLowercase}
  \defaultfontfeatures[\rmfamily]{Ligatures=TeX,Scale=1}
\fi
\usepackage{lmodern}
\ifPDFTeX\else
  % xetex/luatex font selection
\fi
% Use upquote if available, for straight quotes in verbatim environments
\IfFileExists{upquote.sty}{\usepackage{upquote}}{}
\IfFileExists{microtype.sty}{% use microtype if available
  \usepackage[]{microtype}
  \UseMicrotypeSet[protrusion]{basicmath} % disable protrusion for tt fonts
}{}
\makeatletter
\@ifundefined{KOMAClassName}{% if non-KOMA class
  \IfFileExists{parskip.sty}{%
    \usepackage{parskip}
  }{% else
    \setlength{\parindent}{0pt}
    \setlength{\parskip}{6pt plus 2pt minus 1pt}}
}{% if KOMA class
  \KOMAoptions{parskip=half}}
\makeatother
\usepackage{graphicx}
\makeatletter
\newsavebox\pandoc@box
\newcommand*\pandocbounded[1]{% scales image to fit in text height/width
  \sbox\pandoc@box{#1}%
  \Gscale@div\@tempa{\textheight}{\dimexpr\ht\pandoc@box+\dp\pandoc@box\relax}%
  \Gscale@div\@tempb{\linewidth}{\wd\pandoc@box}%
  \ifdim\@tempb\p@<\@tempa\p@\let\@tempa\@tempb\fi% select the smaller of both
  \ifdim\@tempa\p@<\p@\scalebox{\@tempa}{\usebox\pandoc@box}%
  \else\usebox{\pandoc@box}%
  \fi%
}
% Set default figure placement to htbp
\def\fps@figure{htbp}
\makeatother
\setlength{\emergencystretch}{3em} % prevent overfull lines
\providecommand{\tightlist}{%
  \setlength{\itemsep}{0pt}\setlength{\parskip}{0pt}}
\usepackage{bookmark}
\IfFileExists{xurl.sty}{\usepackage{xurl}}{} % add URL line breaks if available
\urlstyle{same}
\hypersetup{
  pdftitle={Base model for queue theory},
  pdfauthor={Optimizados - Markov's Pizza 🍕},
  hidelinks,
  pdfcreator={LaTeX via pandoc}}

\title{Base model for queue theory}
\author{Optimizados - Markov's Pizza 🍕}
\date{2024-12-01}

\begin{document}
\maketitle

\subsection{Descripción del Modelo
Base}\label{descripciuxf3n-del-modelo-base}

Markov's Pizza opera durante 16 horas diarias (desde las 12:00 PM hasta
las 4:00 AM) con dos chefs trabajando simultáneamente para preparar
pizzas para los clientes. Modelaremos la producción de pizzas como un
sistema de colas para analizar la eficiencia y posibles cuellos de
botella en el proceso de elaboración de pizzas.

\subsubsection{Características del Modelo de
Colas}\label{caracteruxedsticas-del-modelo-de-colas}

\begin{itemize}
\tightlist
\item
  \textbf{Tipo de Modelo:} Este escenario se ajusta a un Modelo de Cola
  Multi-Servidor (Cola M/M/2)
\item
  \textbf{M (Llegada):} Distribución exponencial (Poisson) de las
  llegadas de los clientes
\item
  \textbf{M (Servicio):} Distribución exponencial de los tiempos de
  preparación de las pizzas
\item
  \textbf{2 (Servidores):} Dos chefs trabajando simultáneamente
\end{itemize}

\subsubsection{Parámetros Clave a
Considerar}\label{paruxe1metros-clave-a-considerar}

\begin{itemize}
\tightlist
\item
  \textbf{Tasa de Llegada (λ):} Número promedio de pedidos de pizza por
  unidad de tiempo
\item
  \textbf{Tasa de Servicio (μ):} Número promedio de pizzas que un solo
  chef puede preparar por unidad de tiempo, también tomando en cuenta el
  horno compartido entre los chefs
\item
  \textbf{Capacidad del Sistema:} Número máximo de pedidos que pueden
  estar en el sistema (en cola y siendo preparados)
\item
  \textbf{Disciplina de Cola:} Primero en Llegar, Primero en Ser
  Atendido (FCFS)
\end{itemize}

\subsection{\texorpdfstring{Especificaciones del
\href{https://www.fornobravo.com/blog/pizza-oven-size-and-throughput/?srsltid=AfmBOoqfVlB2eAthD4uKjHwTodRZLnPDwe0G9wiqSP5PmHmjBb7xyqpp}{Horno}}{Especificaciones del Horno}}\label{especificaciones-del-horno}

\begin{itemize}
\tightlist
\item
  Tamaño del horno: 56'' horno comercial
\item
  Capacidad: 10 pizzas simultáneamente
\item
  Rendimiento teórico: 12 pizzas por posición por hora (5min por pizza)
\item
  Rendimiento teórico total: 120 pizzas por hora
\item
  Tiempo de horneado: 5 minutos por pizza
\end{itemize}

\subsection{Cálculo de la Tasa de Servicio del Chef y del
Horno}\label{cuxe1lculo-de-la-tasa-de-servicio-del-chef-y-del-horno}

\begin{itemize}
\tightlist
\item
  Tasa de preparación del chef:

  \begin{itemize}
  \tightlist
  \item
    Supongamos que cada chef tarda entre 5 y 7 minutos en preparar una
    pizza.
  \item
    Promedio de 6 minutos por pizza.
  \item
    Tasa de preparación por chef: 10 pizzas por hora (60min / 6min)
  \item
    Tasa de preparación combinada de dos chefs: 20 pizzas por hora
  \end{itemize}
\item
  Tasa de servicio del horno:

  \begin{itemize}
  \tightlist
  \item
    Máximo teórico: 120 pizzas por hora
  \item
    Rendimiento práctico: Supongamos una eficiencia del 80\% (tiempo de
    calentamiento, cambiar fuente de gaz, etc.)
  \item
    Tasa de servicio práctica del horno: 96 pizzas por hora
  \end{itemize}
\end{itemize}

\subsection{Análisis}\label{anuxe1lisis}

\begin{itemize}
\tightlist
\item
  Tiempo de espera promedio de un pedido
\item
  Probabilidad de espera
\item
  Ocio de los chefs
\item
  ``Ocio'' del horno
\item
  Longitud de la cola
\item
  Punto de saturación del sistema
\item
  Analizar posibles mejoras en el proceso tomando en cuenta los costos
  asociados (agregar más chefs, hornos, etc.)
\end{itemize}

\end{document}
